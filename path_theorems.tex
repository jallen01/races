Now that we have defined a coordinate system, we are ready to think about particle paths. All particles in this section have bounded tangential acceleration, $\abs{a_t(t)} \le \bar{a_t}$, for some $\bar{a_t} \ge 0$.

The simplest problem to tackle first is the fastest path from $\bvec{X_0}$ to $\bvec{X_f}$ when centripetal acceleration is unbounded, which means that a particle can change its direction of motion instantaneously. The following theorem makes this notion more precise.

\begin{theorem}\label{thm:theta-choose}
  Let $p$ be a particle with bounded tangential acceleration and $\theta(T_0) = \theta_0$. For all $\theta_f \in [0, \pi]$ and $\epsilon > 0$, there exists a valid path $\gamma(t)$, such that $\abs{\theta(T_0(\gamma) + \epsilon)} \to \theta_f$ as $\epsilon \to 0$.
\end{theorem}
  
\begin{proof}
  Define a a particle's centripetal acceleration to be the following

  \[
      a_c(t) = v(t) \frac{\theta(T_0(\gamma)) - \theta(T_f(\gamma))}{\epsilon}
  \]

  Now, using Equation \ref{eq:theta-diff}

  \[
    \abs{\theta(T_0(\gamma) + \epsilon)} = \theta(T_0(\gamma)) + \int_{t'=T_0(\gamma)}^{T_0(\gamma)+\epsilon} \paren{\frac{d\abs{\phi(t')}}{dt} - \frac{\theta(T_0(\gamma)) - \theta_f}{\epsilon}} dt'
  \]

  Taking the limit as $\epsilon \to 0$

  \[
    \abs{\theta(T_0(\gamma) + \epsilon)} \to \theta_f \quad \text{as} \quad \epsilon \to 0
  \]
\end{proof}

\begin{theorem}\label{thm:theta-min}
  Let $p$ be a particle with bounded tangential acceleration and initial position $X_0$. A path $\gamma_1$ to $\bvec{X_f}$ is a fastest path if, for every other valid path $\gamma_2$ to $\bvec{X_f}$, $\abs{\theta_1(t)} \le \abs{\theta_2(t)}$ and $v_1(t) \le v_2(t)$. The fastest path is unique if all the inequalities are strict.

  Furthermore, if, at some time $t_s \ge T_0(\hat{\gamma})$ along a fastest path $\hat{\gamma}$, $\abs{\theta(t_s)} = 0$, then $\abs{\theta(t)} = 0$ for all $t \ge t_s$. 
\end{theorem}

\begin{proof}
  Referring back to Equation \ref{eq:r-diff}

  \[r(t) = r(T_0) + \int_{t' = T_0}^{t} - v(t') \cos(\abs{\theta(t')}) dt'\]

  Thus, if $\abs{\theta_1(t)} \le \abs{\theta_2(t)}$ and $v_1(t) \le v_2(t)$, then $r_1(t) \le r_2(t)$, and $T_f(\gamma_1) \le T_f(\gamma_2)$, so $\gamma_1$ is a fastest path. If all the inequalities are strict, then $T_f(\gamma_1) < T_f(\gamma_2)$, so every other path is longer, and thus $\gamma_1$ is the unique fastest path.

  The second part of the theorem follows directly from the first part.
\end{proof}

Now we will formalize the natural intuition that straight lines are the fastest paths between points when centripetal acceleration is unbounded.

\begin{theorem}\label{thm:line-proof}
  Let $p$ be a particle with bounded tangential acceleration, initial position $\bvec{X_0}$, and initial velocity $\bvec{v}(t)$. The line segment from $\bvec{X_0}$ to $\bvec{X_f}$ with $a_c(t) = \bar{a_c}$ for all $t \in [T_0(\gamma_1), T_f(\gamma_1)]$ is a fastest path $\hat{\gamma}(t)$ to $\bvec{X_f}$. Furthermore, the fastest path for this system is unique.
\end{theorem}

\begin{proof}
  Because of Theorem \ref{thm:theta-choose}, there is a valid path $\gamma_1(t)$, which has $\theta_1(t) = 0$ for $t \in (T_0(\gamma_1), T_f(\gamma_1)]$.

  If we let $a_{t, 1}(t) = \bar{a_t}$ for all $t \in [T_0(\gamma_1), T_f(\gamma_1)]$, then $v_1(t) = \bar{a_t} t$, so for every other valid path $\gamma_2$

  \begin{align*}
    v_2(t)& = \int_{t'=T_0(\gamma_2}^{t} a_{t, 2}(t') dt'\\
    & \le \bar{a_t} t\\
    &  \le v_1(t)
  \end{align*}

  Thus, all the conditions are satisfied for Theorem \ref{thm:theta-min}, so $\gamma_1(t)$ is a fastest path. The $\theta$ inequality is strict, so the fastest path is unique. Geometrically, since $\theta(t) = 0$, the path is the line segment from $\bvec{X_0}$ to $\bvec{X_f}$.
\end{proof}

% \begin{theorem}\label{thm:line-proof}
%   Given a particle with initial position $\bvec{X_1}$, initial velocity $\bvec{v}(t)$, and infinite centripetal acceleration, the fastest path $\hat{\gamma}(t)$ from $\bvec{X_1}$ to $\bvec{X_2}$ is the line segment from $\bvec{X_1}$ to $\bvec{X_2}$.
% \end{theorem}

% \begin{proof}

% Without loss of generality, we can define a Cartesian coordinate system, where the origin is at $\bvec{X_1}$ and the positive x-axis passes through $\bvec{X_2}$. Let us say that the two points in our Cartesian coordinate system become $(0,0)$ and $(x_2, 0)$ respectively.

% Now let us examine the particle's motion in the $\bhat{x}$ direction. The speed of the particle is the following
% \begin{equation}
%   v_x(t) = \int_0^t a_x(t_1) dt_1
% \end{equation}

% To find the distance $l_x(t)$ traveled in the $\bhat{x}$ direction, we can use the relation:
% \begin{align}
%   l_x(t) &= \int_0^t v_x(t_2) dt_2 \\
%          &= \int_0^t \int_0^{t_1} a_x(t_1) dt_1 dt_2
% \end{align}

% Recall that the tangential acceleration of the particle is bounded by $\bar{a}_{t}$. This means that $a_t(t) \leq \bar{a}_t$ for all $t$. Therefore, we see:
% \begin{align}
% d(t) &\leq \int_0^t \int_0^{t_2} \bar{a}_t dt_1 dt_2\\
%   &= \frac{\bar{a}_t t^2}{2}
% \end{align}

% Thus, in order to travel a distance of $l_x(T_f(\hat{\gamma})) = d' = d(\bvec{X_2}, \bvec{X_1})$, it needs to be the case that $T_f(\hat{\gamma}) \geq \sqrt{\frac{2 d'}{\bar{a}_t}}$. It is possible to travel from $\vec{X_2}$ to $\vec{X_1}$ in time $T_{min} = \sqrt{\frac{2 d'}{\bar{a}_t}}$ if and only if $a_t(t) = \bar{a}_t$ and $\theta(t) = 0$ for all $t \in [T_0(\gamma), T_{min}]$. In other words, the particle must be accelerating at the maximum possible tangential acceleration of $\bar{a}_t$ and it must be accelerating in the straight line direction to $\vec{X_2}$ at all times.

% If the particle travels for time $t < T_{min}$, then it is impossible for the particle to reach $\vec{X_2}$ when starting at $\vec{X_1}$. This is because $p$ cannot reach $\vec{X}_2$ in the $x$ direction when $t < \sqrt{\frac{2 d'}{\bar{a}}}$ because it is impossible for $p$ to reach any point whose $x$-coordinate is $x_2$, so it is obviously impossible to reach $(x_2, 0) = \vec{X_2}$.

% Moreover, if there exists some time $t < T_{min}$ when $\theta(t) \neq 0$, then it is also impossible for the particle to reach $\vec{X_2}$ by time $T_{min}$. Suppose that there are two particles $p_1$ and $p_2$ both accelerating at $a_t(t) = \bar{a}_t$. Imagine $p_1$ satisfies $\theta_{p_1}(t) = 0$ for all $t$ and that $p_2$ satisfies $\theta_{p_2}(t) = 0$ except for times in some interval $[t_0, t_1]$, the particle $p_2$ sets a constant non-zero angle $\zeta_{p_2}(t) \neq 0$ with the $x$-axis for $t \in [t_0, t_1]$. Denote $d_{p}$ as the distance that is left to be traveled by $p$ between times $t_1$ and $T_{min}$. We can find $d_{p_2}$ in terms of $d_{p_1}$ by using trigonometry:

% \begin{eqnarray}
% d_{p_2}^2 = d_{p_1}^2 + \left(\frac{\bar{a}_t (t_1 - t_0)^2}{2} \sin \zeta_{p_2} \right)^2
% \end{eqnarray}

% Since $d_{p_2} \geq 0$ by being a distance, $\sin \zeta_{p_2} \neq 0$ by setting $\zeta_{p_2} > 0$, and $t_1 - t_0 > 0$, we know that $d_{p_2} > d_{p_1}$. This means that at time $T_{min}$ when $p_1$ has reached its destination at $\Vec{X_2}$, particle $p_2$ still has $d_{p_2} - d_{p_1}$ distance left to travel. This means that it is impossible for $\theta(t) > 0$ on the fastest path.

% This means that the fastest path is completed in time $T_f(\hat{\gamma}) = \sqrt{\frac{2 x_2}{\bar{a}}}$. Let us examine the path taken by the point mass $p$ on this fastest path. Recall that $a_t(t) = \bar{a}$ for all $t$ along the fastest path. This means that there was no centripetal acceleration $|a_c| = 0$. In other words, the point mass never turned on its way to reaching the destination point. The only way this could have happened is if it traveled along the $x$ axis in a straight line.

% Thus, we see that the fastest path is the straight line between $\vec{X_1}$ and $\vec{X_2}$.
% \end{proof}

% We can also prove that this fastest path is unique without very much extra work.

% \begin{corollary}
%   The fastest path given by  is unique.
% \end{corollary}
% \begin{proof}
% Since we have starting position $\vec{X_0}$ and initial speed of $0$, the acceleration of the particle uniquely defines a path for the particle.

% There is only a single function of the acceleration $a(t) = \bar{a}_t$ which can be satisfied when the particle is moving along a fastest path, therefore, there is only a single possible fastest path.
% \end{proof}

% We can also investigate how the particle moves along a fastest path. In particular, we can make some statements about the change in $\theta$, the angle between the particle's velocity vector and the straight line between the particle and it's destination.