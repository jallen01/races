Now that we have defined our notation, we are ready to think about the problem of optimal trajectories.

The simplest problem to tackle first, is the optimal trajectory from a starting position, $\bvec{X_0}$ to a final position, $\bvec{X_f}$. Working in a polar coordinate system centered on $X_f$, it is clear, that our goal is achieved when $r = 0$. Furthermore, $\theta(t)$ must be $0$ before reaching $X_f$, otherwise, it will grow exponentially fast as $r(t) \to 0$, which can be seen in equation (\ref{eq:r-deriv}).

\begin{lemma}
If $a_t=0$ and $a_c \le a_{c,max}$, then the minimum radius of curvature of a particle's trajectory is given by
\begin{equation}
R_{min} = \frac{v^2}{a_{c,max}}
\end{equation}
\end{lemma}

\begin{proof}
This comes directly from the definition of radius of curvature 

\[
R = \frac{v^2}{a_c}
\]

and the fact that 

\[
a_c \le a_{c,max} \to \frac{1}{a_c} \ge \frac{1}{a_{c,max}}
\]
\end{proof}

\begin{theorem}
A 2-dimensional particle, $p$, traveling along a valid path, $\hat{\gamma(t)}$, from $X_1$ to $X_2$ has $\frac{d\abs{\theta(t)}}{dt} < 0$ for all $t \in [0, T_{f, \gamma}]$, unless $\norm{X_1 - X_2} < 2 R_{min}$.
\end{theorem}

\begin{proof}
Since $\theta(T_{f,\gamma}) = 0$, then if $\frac{d\abs{\theta(t)}}{dt} > 0$ at some point along the path, then $\theta{T_{f,\gamma}} < \theta{t_0} < \theta{t_c}$, so [TODO]
\end{proof}

\begin{theorem}
Given a 2-dimensional particle, $p$, it is always optimal to minimize $\frac{d\abs{\theta(t)}}{dt}$ at all $t \in [0, T_{\gamma}]$.
\end{theorem}

\begin{proof}
[TODO]
\end{proof}

\begin{theorem}
Given a 2-dimensional particle, $p$, with bounded centripetal acceleration, $a_c \le a_{c,max}$ and zero tangential acceleration, every valid optimal path, $\hat{\gamma}(t)$ between two points $X_1$ and $X_2$, has a radius of curvature of $0$ or $a_{c,max}$ at every point along it.
\end{theorem}

\begin{proof}
[TODO]
\end{proof}

\begin{corollary}\label{corr:optimal-path-fix-s}
Given a 2-dimensional particle, $p$, with bounded centripetal acceleration, $a_c \le a_{c,max}$ and zero tangential acceleration, every valid optimal trajectory is constructed from line segments and circular sections of radius $R_{min}$, that are tangent at their intersections.
\end{corollary}

\begin{proof}
[TODO]
\end{proof}

% -----------------------------------------------------------------------------

Now we will show that the fastest path between two points goes in a straight line. This will formalize the natural intuition that straight lines are the fastest path when there are no obstacles between the start and finish position.

\begin{theorem}
  Given a 2-dimensional particle $p$ whose initial position is $\bvec{X_1}$ and moves with bounded tangential acceleration, $a_t \le a_{t,max}$, and infinite centripetal acceleration, the fastest path $\hat{\gamma}(t)$ which $p$ can trace from $\bvec{X_1}$ to $\bvec{X_2}$ lies on the line segment from $\bvec{X_1}$ to $\bvec{X_2}$:
\end{theorem}

\begin{proof} 

Without loss of generality, we can define a cartesian coordinate system, where the origin is at $\bvec{X_1}$ and the positive x-axis passes through $\bvec{X_2}$.

Now let us examine the particle's motion in the $\bhat{x}$ direction. The speed of the particle is the following
\begin{equation}
  v_x(t) = \int_0^t a_x(t_1) dt_1
\end{equation}

To find the distance $l_x(t)$ travelled in the $\bhat{x}$ direction, we can use the relation:
\begin{align}
  d(t) &= \int_0^t v_x(t_2) dt_2 \\
  &= \int_0^t \int_0^t a_x(t_1) dt_1 dt_2
\end{align}

Recall that the tangential acceleration of the point mass $p$ is bounded by $a_{t,max}$. This means that $a_t(t) \leq \bar{a}$ for all $t$. Therefore, we see:
\begin{align}
  d(t) &\leq \int_0^t \int_0^t \bar{a} dt_1 dt_2\\
  &= \frac{\bar{a} t^2}{2}
\end{align}

Thus, in order to travel a distance of $d(T_f) = x_2$, it needs to be the case that $T_f \geq \sqrt{\frac{2 x_2}{\bar{a}}}$. Moreover, equality holds if and only if $a_t(t) = \bar{a}$ for all $t \in [0, T_f(\gamma)]$.

If the point mass travels for time $t < \sqrt{\frac{2 x_2}{\bar{a}}}$, then it is impossible for the point mass to reach $(x_2, 0)$ when starting at $(0,0)$. This is because $p$ cannot reach $(x_2, 0)$ in the $x$ direction when $t < \sqrt{\frac{2 x_2}{\bar{a}}}$ and any acceleration in the $y$ direction would not enable this either.

This means that the fastest path is completed in time $T_f(\hat{\gamma}) = \sqrt{\frac{2 x_2}{\bar{a}}}$. Let us examine the path taken by the point mass $p$ on this fastest path. Recall that $a_t(t) = \bar{a}$ for all $t$ along the fastest path. This means that there was no centripetal acceleration $|a_c| = 0$. In other words, the point mass never turned on its way to reaching the destination point. The only way this could have happened is if it travelled along the $x$ axis in a straight line.

Now, we have seen that the fastest path in the transformed coordinates travels exactly on the $x$ axis so that $y = 0$ anywhere along the fastest path. Notice, however, that the $x$ axis in the transformed coordinates is given exactly by the following line:
\begin{equation}
  y = \frac{y_2 - y_1}{x_2 - x_1} x + y_1
\end{equation}

Thus, we see that the fastest path in the original coordinates follows the above equation, which is what we wanted to show.
\end{proof}

\begin{corollary}
  Given points $(x_1, y_1), (x_2, y_2) \in \mathrm{R}^2$ and a particle $p$ whose initial position is $(x_1, y_1)$ which moves with acceleration bounded by $\bar{a}$, the fastest path $\hat{\gamma}(t)$ which $p$ can trace from $(x_1, y_1)$ to $(x_2, y_2)$ is unique.
\end{corollary}
\proof We have already shown that any fastest path between $(x_1, y_1)$ and $(x_2, y_2)$ follows the straight line given by $y = \frac{y_2 - y_1}{x_2 - x_1} x + y_1$. Moreover, we showed that when travelling along the fastest path, the particle must have acceleration along the straight line of $\bar{a}$. Since we have starting position $(x_1, y_1)$ and initial speed of $0$, the acceleration of the particle $a(t)$ uniquely defines a path for the particle.

There is only a single function $a(t) = \bar{a}$ which the acceleration can satisfy when the particle is moving along a fastest path, therefore, there is only a single possible fastest path.
\qed

% -----------------------------------------------------------------------------

