% -----------------------------------------------------------------------------

\begin{theorem}
  Given points, $X_1, X_2 \in \R^2$ and a particle $p$ initially at $X_1$ with infinite centripetal acceleration bounded tangential acceleration, $\abs{a_t} \le \bar{a}$, then the fastest path $\hat{\gamma}(t)$ which $p$ can trace from $X_1$ to $X_2$ is the line segment from $X_1$ to $X_2$.
\end{theorem}

\proof 

Let's transform the problem. We can reset our coordinate axes so that $(x_1, y_1)$ is set to the origin and $(x_2, y_2)$ is on the x-axis. In this new coordinate system, we have transformed the following:
\begin{eqnarray}
  (x_1, y_1) &\to& (0,0) \\
  (x_2, y_2) &\to& (x_2', 0)
\end{eqnarray}

For convenience of notation, we will now refer to $x_2'$ as $x_2$.

Now let us examine the particle's motion in the $x$ direction. Let $a_t(t)$ be the tangential acceleration at time $t$ in the $x$ direction. Then we can obtain the speed of the particle $s(t)$ at time $t$ in the $x$ direction like so:
\begin{eqnarray}
  s(t) = \int_0^t a_t(t_1) dt_1
\end{eqnarray}

To find the distance $d(t)$ travelled up to time $t$ in the $x$ direction, we can use the relation:
\begin{eqnarray}
  d(t) &=& \int_0^t s(t_2) dt_2 \\
       &=& \int_0^t \int_0^t a_t(t_1) dt_1 dt_2
\end{eqnarray}

Recall that the acceleration of the particle $p$ is bounded by $\bar{a}$. This means that $a_t(t) \leq \bar{a}$ for all $t$. Therefore, we see:
\begin{eqnarray}
  d(t) &\leq& \int_0^t \int_0^t \bar{a} dt_1 dt_2 \\
       &=& \frac{\bar{a} t^2}{2}
\end{eqnarray}

Thus, in order to travel a distance of $d(T_f) = x_2$, it needs to be the case that $T_f \geq \sqrt{\frac{2 x_2}{\bar{a}}}$. Moreover, equality holds if and only if $a_t(t) = \bar{a}$ for all $t \in [0, T_f(\gamma)]$.

If the particle travels for time $t < \sqrt{\frac{2 x_2}{\bar{a}}}$, then it is impossible for the particle to reach $(x_2, 0)$ when starting at $(0,0)$. This is because $p$ cannot reach $(x_2, 0)$ in the $x$ direction when $t < \sqrt{\frac{2 x_2}{\bar{a}}}$ and any acceleration in the $y$ direction would not enable this either.

This means that the fastest path is completed in time $T_f(\hat{\gamma}) = \sqrt{\frac{2 x_2}{\bar{a}}}$. Let us examine the path taken by the particle $p$ on this fastest path. Recall that $a_t(t) = \bar{a}$ for all $t$ along the fastest path. This means that there was no centripetal acceleration $|a_c| = 0$. In other words, the particle never turned on its way to reaching the destination point. The only way this could have happened is if it travelled along the $x$ axis in a straight line.

Now, we have seen that the fastest path in the transformed coordinates travels exactly on the $x$ axis so that $y = 0$ anywhere along the fastest path. Notice, however, that the $x$ axis in the transformed coordinates is given exactly by the following line:
\begin{eqnarray}
  y = \frac{y_2 - y_1}{x_2 - x_1} x + y_1
\end{eqnarray}

Thus, we see that the fastest path in the original coordinates follows the above equation, which is what we wanted to show.
\qed

% -----------------------------------------------------------------------------

\begin{corollary}
  The fastest path between two points $(x_1, y_1), (x_2, y_2) \in \R^2$ is unique.
\end{corollary}

% -----------------------------------------------------------------------------

\includefig{polar_param.eps}{Particle parameters in polar coordinates. B is center of coordinate system, and p is the particle.}{polar-param}

\begin{lemma}
  For a particle, $p$,
  \begin{align*}
    \frac{dr(t)}{dt} = -s(t) \, \cos\paren{\phi(t) + \theta(t)}\\
    \frac{d\phi(t)}{dt} = -s(t) \, \sin\paren{\phi(t) + \theta(t)}
  \end{align*}
\end{lemma}

\begin{lemma}\label{theta-deriv}
  For a particle, $p$,
  \[
  \frac{d\theta(t)}{dt} = -\frac{a_c(t)}{s(t)}
  \]
\end{lemma}

\begin{lemma}
  For a particle, $p$, with bounded centripetal and tangential acceleration, than the functions $\phi(t)$ and $\theta(t)$ are continuous. Furthermore, for two times $t_1$ and $t_2$, s.t. $t_1 \le t_2$, $\phi(t_1) + \theta(t_1) > a$ and  
\end{lemma}

\proof

The proof follows directly from the fact that the derivatives of $\phi(t)$ and $\theta(t)$ exist: \ref{theta-deriv}).

\begin{lemma}
  For a particle, $p$, with nonzero speed, and no centripetal acceleration for $t \ge t_0$, then
  \[
    \begin{cases}
      \phi(t) \to \theta(t_0) \quad \text{as} \quad t \to \infty \qquad &\text{if} \;\; \theta(t_0) > 0\\
      \phi(t) = 0 \quad \text{for} \quad t \ge t_0 \qquad &\text{if} \;\; \theta(t_0) = 0
    \end{cases}
  \]
\end{lemma}