\begin{lemma}
If $a_t=0$ and $\norm{a_c} \le a_{c,max}$, then the minimum radius of curvature of a point's trajectory is given by
\begin{equation}
R_{min} = \frac{v^2}{a_{c,max}}
\end{equation}
\end{lemma}

\begin{proof}
This comes directly from the definition of radius of curvature as 

\[
R = \frac{v^2}{a_c}
\]
\end{proof}

\begin{theorem}
For a particle, $p$, traveling along an optimal path, $\hat{\gamma}(t)$ between two points $X_1$ and $X_2$, 
\end{theorem}

\begin{lemma}
Every optimal trajectory is constructed from line segments and circular sections of radius $R_{min}$. Every point along an optimal trajectory where $a_t=0$ has $R = \infty$ or $R = R_{min}$.
\end{lemma}

\proof

If we look at a position, p, on a trajectory. We can align a rectangular coordinate system with this point, such that $\hat{y} = \hat{v}$.

\begin{align}
a_c& = \ddot{r} - r \dot{\phi}^2\\
%
a_t& = \frac{1}{r} \frac{d}{dt} \paren{r^2 \dot{\phi}}\\
%
0& = \frac{1}{r} \frac{d}{dt} \paren{r^2 \dot{\phi}}\\
%
r^2 \dot{\phi}& = constant
\end{align}

\qed

% -----------------------------------------------------------------------------

\begin{theorem}
  Given points $(x_1, y_1), (x_2, y_2) \in \mathrm{R}^2$ and a particle $p$ whose initial position is $(x_1, y_1)$ which moves with acceleration bounded by $\bar{a}$, the fastest path $\hat{\gamma}(t)$ which $p$ can trace from $(x_1, y_1)$ to $(x_2, y_2)$ follows the straight line where all coordinates $(x,y)$ on the straight line are given by:
  \begin{eqnarray}
    y = \frac{y_2 - y_1}{x_2 - x_1} x + y_1
  \end{eqnarray}
\end{theorem}
\proof Let's transform the problem. We can reset our coordinate axes so that $(x_1, y_1)$ is set to the origin and $(x_2, y_2)$ is on the x-axis. In this new coordinate system, we have transformed the following:
\begin{eqnarray}
  (x_1, y_1) &\to& (0,0) \\
  (x_2, y_2) &\to& (x_2', 0)
\end{eqnarray}

For convenience of notation, we will now refer to $x_2'$ as $x_2$.

Now let us examine the particle's motion in the $x$ direction. Let $a_t(t)$ be the tangential acceleration at time $t$ in the $x$ direction. Then we can obtain the speed of the particle $s(t)$ at time $t$ in the $x$ direction like so:
\begin{eqnarray}
  s(t) = \int_0^t a_t(t_1) dt_1
\end{eqnarray}

To find the distance $d(t)$ travelled up to time $t$ in the $x$ direction, we can use the relation:
\begin{eqnarray}
  d(t) &=& \int_0^t s(t_2) dt_2 \\
       &=& \int_0^t \int_0^t a_t(t_1) dt_1 dt_2
\end{eqnarray}

Recall that the acceleration of the point mass $p$ is bounded by $\bar{a}$. This means that $a_t(t) \leq \bar{a}$ for all $t$. Therefore, we see:
\begin{eqnarray}
  d(t) &\leq& \int_0^t \int_0^t \bar{a} dt_1 dt_2 \\
       &=& \frac{\bar{a} t^2}{2}
\end{eqnarray}

Thus, in order to travel a distance of $d(T_f) = x_2$, it needs to be the case that $T_f \geq \sqrt{\frac{2 x_2}{\bar{a}}}$. Moreover, equality holds if and only if $a_t(t) = \bar{a}$ for all $t \in [0, T_f(\gamma)]$.

If the point mass travels for time $t < \sqrt{\frac{2 x_2}{\bar{a}}}$, then it is impossible for the point mass to reach $(x_2, 0)$ when starting at $(0,0)$. This is because $p$ cannot reach $(x_2, 0)$ in the $x$ direction when $t < \sqrt{\frac{2 x_2}{\bar{a}}}$ and any acceleration in the $y$ direction would not enable this either.

This means that the fastest path is completed in time $T_f(\hat{\gamma}) = \sqrt{\frac{2 x_2}{\bar{a}}}$. Let us examine the path taken by the point mass $p$ on this fastest path. Recall that $a_t(t) = \bar{a}$ for all $t$ along the fastest path. This means that there was no centripetal acceleration $|a_c| = 0$. In other words, the point mass never turned on its way to reaching the destination point. The only way this could have happened is if it travelled along the $x$ axis in a straight line.

Now, we have seen that the fastest path in the transformed coordinates travels exactly on the $x$ axis so that $y = 0$ anywhere along the fastest path. Notice, however, that the $x$ axis in the transformed coordinates is given exactly by the following line:
\begin{eqnarray}
  y = \frac{y_2 - y_1}{x_2 - x_1} x + y_1
\end{eqnarray}

Thus, we see that the fastest path in the original coordinates follows the above equation, which is what we wanted to show.
\qed

\begin{corollary}
  Given points $(x_1, y_1), (x_2, y_2) \in \mathrm{R}^2$ and a particle $p$ whose initial position is $(x_1, y_1)$ which moves with acceleration bounded by $\bar{a}$, the fastest path $\hat{\gamma}(t)$ which $p$ can trace from $(x_1, y_1)$ to $(x_2, y_2)$ is unique.
\end{corollary}
\proof We have already shown that any fastest path between $(x_1, y_1)$ and $(x_2, y_2)$ follows the straight line given by $y = \frac{y_2 - y_1}{x_2 - x_1} x + y_1$. Moreover, we showed that when travelling along the fastest path, the particle must have acceleration along the straight line of $\bar{a}$. Since we have starting position $(x_1, y_1)$ and initial speed of $0$, the acceleration of the particle $a(t)$ uniquely defines a path for the particle.

There is only a single function $a(t) = \bar{a}$ which the acceleration can satisfy when the particle is moving along a fastest path, therefore, there is only a single possible fastest path.
\qed

% -----------------------------------------------------------------------------

