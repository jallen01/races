In this paper, we have formalized many of the things that intuition would tell us. Namely, we have shown that a straight line is the fastest way to get between two points. This fact is unsurprising because of the fact that acceleration in a single direction (toward the finish point) is the least wasteful means of getting to the finish.

We developed a number of simple lemmas governing the motion of a volume-less particle with bounded acceleration. We also showed that when a particle has bounded accleration, the optimal strategy for getting to another point as fast as possible is to minimize the angle that the particle's velocity vector makes with the ending point.

\subsection{Future Work}

Future work should first involve a rigorous proof of . Furthermore, the special case of optimal paths for when the particle is \"too close\" to the final point, should be researched. This special case is very interesting, because it involves moving away from the final point for a period of time, and the shape of the optimal path does not appear to be intuitive.

To make the problem more general, one could extend the problem to include particles moving with varying speed. Equations \ref{eq:r-diff} - \ref{eq:r-diff-2} still apply in this case, but the analysis is more complex in this case. 
