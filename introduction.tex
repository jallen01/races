The problem of finding fastest paths between two points (let's call them A and B) is interesting because many attributes of a fastest path are intuitively obvious, but rigorously proving them is much more difficult. One's first assumption would be that the fastest path between A and B would be to travel in a straight line from A to B. This would be true in the simple case where the turning rate is not constrained, but it is not true in the general case when a particle has a constrained turning rate and starts with some nonzero velocity which is not directed towards B. In the general case, it is easier to turn towards B the further one is away from it.

\subsection{The Problem (more precise terms will be defined later)}

Our problem is the following: given a particle in $\R^2$ starting at position A with some initial velocity, what is the path that the particle must take to reach a final position B such that the time spent traveling from A to B is minimized? The particle can have restrictions on its position, velocity, and acceleration. Our model problem has fixed speed and a bounded turning rate.

\subsection{Motivation}

The problem of finding a fastest path for a particle with constant speed is useful because it is identical to the problem of finding a shortest path given constraints on the radius of curvature of the path, which appears in fields such as physics and computational geometry.

Intuitively, one could imagine a race car driver navigating a racetrack with constant speed, as a physical example of the problem. In fact, this example directly motivates our problem, because a race car driver has a limited turning rate which is created by friction between the wheels and the ground.

Understanding how to find fastest paths for a particle with constant speed will also provide insight into solving even more complicated problems which may have varying speed and more dimensions such as spacecraft trajectories.

\subsection{Outline}

We first present a modified polar coordinate system to represent the particle's motion around a point and then derive a set of differential equations governing the motion of the particle in this coordinate system. This allows us to devise some simple lemmas about the motion of the particle, and draw some conclusions about fastest paths.
