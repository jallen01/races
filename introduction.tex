The problem of finding optimal paths between points for a particle with fixed speed is interesting because many attributes of an optimal path are intuitively obvious, but rigorously proving them is much more difficult. One's first assumption would be that the optimal path between points A and B, would be to travel in a straight line from point A to B. This would be true in the simple case where the turning rate is not constrained, but it is not true in the general case when a particle has a constrained turning rate and starts with some nonzero velocity not directed towards B. 

\subsection{The Problem}

Given a particle that starts at a position in $\R^2$, has initial velocity, and has constrained acceleration, what is the path in $\R^2$ for the particle to take to reach a final position in $\R^2$, such that the time spent traveling from the starting position to the ending position is minimized. The particle can have restrictions on its position, velocity, and acceleration.

\subsection{Motivation}

The problem of finding a fastest path for a particle with constant speed is usefull because it is identical to the problem of finding a shortest path in $\R^2$ given constraints on the radius of curvature of the path, which appears in fields such as physics and computational geometry. One could imagine a race car driver navigating a racetrack with constant speed, as a physical example of the problem.

This example directly motivates our problem, because a racecar driver has a turning rate limited by friction.

Understanding how to find optimal paths for a particle with constant speed moving in $\R^2$ will provide insight into solving even more complicated problems with varying speed and more dimensions such as spacecraft trajectories.

 

\subsection{Outline}

We first present a modified polar coordinate representation of the particle's position around a point and derive a set of differential equations governing the motion of the particle in this coordinate system. This allows us to devise some simple lemmas about the motion of the particle.