The problem of finding optimal paths between points for a particle with fixed speed is interesting because many attributes of an optimal path are intuitively obvious, but rigorously proving them is much more difficult. One's first assumption would be that the optimal path between points A and B, would be to travel in a straight line from point A to B. This would be true in the simple case where the turning rate is not constrained, but it is not true in the general case when a particle has a constrained turning rate and starts with some nonzero velocity not directed towards B. 


\subsection{Motivation}

In physics and computational geometry the problem of finding the shortest path given constraints on the radius of curvature.

There are many instances where finding optimal paths is important. For instance, a race car driver wants to know the optimal way to get from one point to another so that he can win his race. For another example, imagine you are sending a spacecraft to a particular destination in space and would like the optimal means of getting there.

The space example directly motivates our model problem. On a spacecraft, there is a limited amount of acceleration that is possible (provided by thrusters). One would potentially like to navigate to a planet while moving around obstacles. 

Understanding how to find optimal trajectories will provide greater insight into solving problems like these.

\subsection{Problem Description}

Given a particle, $p$, moving in $\R^2$, we set constraints on its position, velocity . The particle has a starting position and velocity, and must navigate in $\R^2$ to reach a final position. The goal is to minimize the total time spent navigating to the final position. 

\subsection{Outline}

We first present a modified polar coordinate representation of the particle's position around a point and derive a set of differential equations governing the motion of the particle in this coordinate system. This allows us to devise some simple lemmas about the motion of the particle.