In this paper, we shall examine optimal paths for a particle.

Our model problem is a particle moving in $\R^2$ with constrained acceleration. The particle must navigate around cones (points in $\R^2$) to reach a final position. The goal is to minimize the total time spend navigating to the final position. 

We first present a polar coordinate representation of a particle's position around a point and derive a set of differential equations governing the motion of the particle in this coordinate system. This allows us to devise some simple lemmas about the motion of the particle.

After developing an intuition for particles and particle paths, we then move on to the more complicated problem of deriving governing rules for optimal paths between points.

Finally, we tackle the problem of finding an optimal path around cones.

\subsection{Motivation}

The problem of finding an optimal trajectory is interesting because of its applications in physics. There are many instances where finding optimal paths is important. For instance, a race car driver wants to know the fastest way to get from one point to another so that he can win his race. For another example, imagine you are sending a spacecraft to a particular destination in space and would like the fastest means of getting there.

The space example directly motivates our model problem. On a spacecraft, there is a limited amount of acceleration that is possible (provided by thrusters). One would potentially like to navigate to a planet while moving around obstacles. 

Understanding how to find optimal trajectories will provide greater insight into solving problems like these.
