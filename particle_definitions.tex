In this section, we will provide some basic definitions about particles and paths. These will lay the groundwork for thinking about optimal paths.

\begin{definition}
A n-dimensional path $\gamma(t): \mathrm{R} \to \mathrm{R}^n$ is a function which maps a time $t \in \R$, $t \in [0, T_{f, \gamma}]$, to a position $\bvec{X} \in \mathrm{R}^n$. 
\end{definition}

\begin{definition}
  A n-dimensional particle, $p$, is an object with zero volume that travels along a n-dimensional path. Traveling along a path means that the particle is at position $\lambda(t)$ at time $t$, for all $t \in [0, T_{f,\lambda}]$. The particle may have conditions on its position, velocity, and acceleration in $\R^n$.
\end{definition}

\begin{definition}
  A valid path $\gamma(t)$ for a particle $p$ is a path along which $p$ can travel, such that all its conditions are satisfied at every point along the path.
\end{definition}

\begin{definition}
  A path between two points, $\bvec{X_1}$ and $\bvec{X_2}$ is a path, $\gamma(t)$, where $\gamma(0) = \bvec{X_1}$ and $\gamma(T_{f, \gamma}) = \bvec{X_2}$.
\end{definition}

\begin{definition}
  For a given particle, $p$, a fastest path, $\hat{\gamma}(t)$, between two points, $\bvec{X_1}$ and $\bvec{X_2}$, is a valid path such that $T_{f,\hat{\gamma}} \leq T_{f,\gamma}$ for all valid paths, $\gamma(t)$, between $\bvec{X_1}$ and $\bvec{X_2}$.
\end{definition}

\begin{definition}
  A particle's velocity is defined as 

  \[\bvec{v} \coloneqq \frac{d\bvec{X}}{dt}\]

  Furthermore, a particle's speed is $v$, and its direction of motion is $\bhat{v}$

\end{definition}

\begin{definition}
  A particle's acceleration is defined as 
  
  \[\bvec{a} \coloneqq \frac{d\bvec{v}}{dt}\]
\end{definition}

\begin{definition}
  The centripetal acceleration, $\bvec{a_c}$, of a particle, $p$, is the component of its acceleration in the direction perpendicular to its direction of motion.

  In rectangular coordinates, the sign of $a_c$ is defined to be the sign of the projection of $\bhat{a_c}$ onto $\bhat{x}$. 

  In polar coordinates, the sign of $a_c$ is defined to be the sign of the projection of $\bhat{a_c}$ onto $\bhat{r}$.
\end{definition}

\begin{definition}
  The tangential acceleration, $\bvec{a_t}$, of a particle, $p$, is the component of the acceleration of $p$ in its direction of motion, $\bhat{v}$.

  \[
  a_t = \frac{dv}{dt}
  \]
\end{definition}

% -----------------------------------------------------------------------------

\subsection{Particle Motion in Polar Coordinates}

Unless otherwise specified, the motion of 2-dimensional particles in this paper will be described in polar coordinates, along with an extra parameter $\theta$ as is shown in Figure \ref{fig:polar-param}.

\includefig{polar_param.eps}{A particle moving in 2-dimensional polar coordinate system centered at B}{polar-param}

$\theta(t), \phi(t) \in [-\pi, \pi]$. From now on, $\abs{\theta}$ and $\abs{\phi}$ will be used, since there is a symmetry in the system about $\bhat{r}$.

\begin{lemma}
The time derivative of $\theta$ is given by
\[
\frac{d\abs{\theta(t)}}{dt} = \frac{a_c}{v}
\]
\end{lemma}

\begin{proof}

If we look at a point, $p$, subject to only centripetal acceleration, $a_c$, the change in $\bvec{v}$ over an infinitesimal time, $dt$, is shown in Figure \ref{fig:theta-deriv} (the two vectors, $\bvec{v}$ and $\bvec{v}+\bvec{dv}$, are superimposed). 

\includefig{theta_deriv.eps}{Tangential acceleration}{theta-deriv}

From the definition of $a_c$

\[
\frac{dv}{dt} = a_c
\]

Since $dv = d\abs{\theta} v$, then
\[
\frac{d\abs{\theta}}{dt} = \frac{a_c}{v}
\]
\end{proof}

The proof of the lemma in the case where $a_t$ is also nonzero is very similar, since the component of $\bvec{dv}$ in the direction $\bhat{v}$ is negligible compared to $v$.

Returning again to Figure \ref{fig:polar-param}, the following equations can be derived

\begin{align}
  \frac{dr(t)}{dt}& = -v(t) \, \cos\paren{\abs{\theta(t)}}\label{eq:r-deriv}\\
  \frac{d\abs{\phi(t)}}{dt}& = \frac{v(t)}{r(t)} \, \sin\paren{\abs{\theta(t)}}\label{eq:phi-deriv}\\
  \frac{d\abs{\theta(t)}}{dt}& = \frac{d\abs{\phi(t)}}{dt} - \frac{a_c(t)}{v(t)}\\
  &= \frac{v(t)}{r(t)} \, \sin\paren{\abs{\theta(t)}} - \frac{a_c(t)}{v(t)}\label{eq:theta-deriv}\\
  \intertext{Applying the chain rule to (\ref{eq:r-deriv})}
  \frac{d}{dt} \frac{dr(t)}{dt}& = -\frac{dv(t)}{dt} \cos\paren{\abs{\theta(t)}} + v(t) \sin\paren{\abs{\theta(t)}} \frac{d\abs{\theta(t)}}{dt}\\
  &\begin{aligned}
    = -\frac{dv(t)}{dt} \cos\paren{\abs{\theta(t)}} &+ \frac{v(t)^2}{r(t)} \sin^2\paren{\abs{\theta(t)}}\\
    &\qquad - v(t) \sin\paren{\abs{\theta(t)}} \frac{a_c(t)}{v(t)}\label{eq:r-deriv-2}
  \end{aligned}
\end{align}

\begin{lemma}
  For a particle, $p$, with bounded centripetal and tangential acceleration, then: 

  1. The function $\abs{\theta(t)}$ is continuous. 

  2. Given $t_1, t_2 \in \R^2$, s.t. $t_1 < t_2$, $\theta(t_1) > a$ and $\theta(t_2) < a$, then $\theta(t_c) = a$ for some $t_c \in [t_1, t_2]$.
\end{lemma}

\begin{proof}
First off, it should be noted that a solution to (\ref{eq:theta-deriv}) exists because the right hand side is Lipschitz continuous. The proof of 1.\ follows directly from the fact that the derivative of $\abs{\theta(t)}$ exists and is bounded. 2.\ is just a restatement of the intermediate value theorem.
\end{proof}

\begin{lemma}\label{lemma:polar-assymptote}
  For a particle, $p$, with nonzero speed, and zero centripetal acceleration for $t \ge t_0$, then
  \[
    \begin{cases}
      \abs{\theta(t)} \to \pi \quad \text{as} \quad t \to \infty \qquad &\text{if} \;\; \theta(t_0) > 0\\
      \abs{\theta(t)} = 0 \quad \text{for all} \quad t \ge t_0 \qquad &\text{if} \;\; \theta(t_0) = 0
    \end{cases}
  \]

    Furthermore,

  \[
    \begin{cases}
      r(t) \to \infty \quad \text{as} \quad t \to \infty \qquad &\text{if} \;\; \theta(t_0) > 0\\
      r(t) \to 0 \quad \text{as} \quad t \ge t_0 \qquad &\text{if} \;\; \theta(t_0) = 0
    \end{cases}
  \]
\end{lemma}

\begin{proof}

For the first case, $\frac{d\abs{\theta(t)}}{dt} > 0$, except when $\abs{\theta(t)} = 0, \pi$, which means that $\theta(t)$ is monotonically increasing. Since $\phi(t)$ is bounded above py $\pi$, $\phi(t) \to \pi \;\; \text{as} \;\; \infty$.

The second case is the easiest to check. If we plug $\theta(t_0) = 0$ into \ref{eq:theta-deriv}, we get

\begin{align*}
\frac{d\abs{\theta(t)}}{dt} = 0 \qquad \text{for} \;\; t \ge t_0\\
\intertext{So}
\abs{\theta(t)} = 0 \qquad \text{for} \;\; t \ge t_0
\end{align*}

[TODO]

\end{proof}