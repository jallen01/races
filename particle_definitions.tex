In this section, we will make some basic definitions about particles and paths. These will lay the groundwork for thinking about optimal paths. We will begin by defining the description of a particle and then define various types of paths that a particle can take.

\begin{definition}
A n-dimensional path $\gamma(t): \mathrm{R} \to \mathrm{R}^n$ is a function which maps a time $t \in \R$, $t \in [0, T_{f, \gamma}]$, to a position $\bvec{X} \in \mathrm{R}^n$. 
\end{definition}

\begin{definition}
  A n-dimensional particle, $p$, is an object with zero volume that travels along a n-dimensional path. The particle may have conditions on its position, velocity, and acceleration in $\R^n$.
\end{definition}

\begin{definition}
  A valid path $\gamma(t)$ for a particle $p$ is a path such that all conditions on the particle are satisfied at every point along the path.
\end{definition}

\begin{definition}
  A path between two points, $\bvec{X_1}$ and $\bvec{X_2}$ is a path, $\gamma(t)$ where $\gamma(0) = \bvec{X_1}$ and $\gamma(T_{f, \gamma}) = \bvec{X_2}$.
\end{definition}

\begin{definition}
  For a given particle, $p$, a fastest path, $\hat{\gamma}(t)$, between two points, $\bvec{X_1}$ and $\bvec{X_2}$, is a valid path such that $T_f(\hat{\gamma}) \leq T_f(\gamma)$ for all valid paths, $\gamma(t)$, between $\bvec{X_1}$ and $\bvec{X_2}$.
\end{definition}

\begin{definition}
  The centripetal acceleration, $\bvec{a_c}$, of a particle, $p$, is the component of the acceleration of $p$ perpendicular to its direction of motion, $\bhat{v}$. The sign of $a_c$ is difined as the sign of the projection of $\bhat{a_c}$ onto $\bhat{r}$.
\end{definition}

\begin{definition}
  The tangential acceleration, $\bvec{a_t}$, of a particle, $p$, is the component of the acceleration of $p$ in its direction of motion, $\bhat{v}$.
\end{definition}

% -----------------------------------------------------------------------------

\subsection{Particle Motion in Polar Coordinates}

The motion of the 2-dimensional particles in this paper will typically be described in the polar coordinate system, shown in figure \ref{fig:polar-param}.

\includefig{polar_param.eps}{A particle moving in 2-dimensional polar coordinate system centered at B}{polar-param}

\begin{lemma}
The time derivative of $\theta$ is given by
\[
\frac{d\beta}{dt} = \frac{a_c}{s}
\]
\end{lemma}

\begin{proof}

\includefig{theta_deriv.eps}{Tangential acceleration}{theta-deriv}

If we look at a point, $p$, subject to only a centripetal acceleration, $a_c$, the change in $\bvec{v}$ over an infinitesimal time, $dt$, is shown in the figure \ref{fig:theta-deriv} (the two vectors, $\bvec{v}$ and $\bvec{v}+\bvec{dv}$, are superimposed). From the definition of $a_c$

\[
\frac{dv}{dt} = a_c
\]

So...
\[
\frac{d\beta}{dt} = \frac{a_c}{s}
\]
\end{proof}

The proof of the lemma in the case where $a_t$ is also nonzero is very similar, since the component of $\bvec{dv}$ in the direction $\bhat{v}$ is negligeable compared to $v$.

Returning again to figure \ref{fig:polar-param}, the following equations can be derived

\begin{align}
  \frac{dr(t)}{dt}& = -s(t) \, \cos\paren{\theta(t)}\\
  \frac{d\phi(t)}{dt}& = s(t) \, \sin\paren{\theta(t)}\\
  \frac{d\theta(t)}{dt}& = \frac{d\phi(t)}{dt} - \frac{a_c(t)}{s(t)}\label{eq:theta-deriv}\\
  \frac{d}{dt} \frac{dr(t)}{dt}& = -\frac{ds(t)}{dt} \cos\paren{\theta(t)} + s(t) \sin\paren{\theta(t)} \frac{d\theta(t)}{dt}
\end{align}

\begin{lemma}
  For a particle, $p$, with bounded centripetal and tangential acceleration, then: 

  (1) The functions $\phi(t)$ and $\theta(t)$ are continuous. 

  (2) For two times $t_1$ and $t_2$, s.t. $t_1 < t_2$, $\theta(t_1) > a$ and $\theta(t_2) < a$, then $\theta(t_c) = a$ for some $t_c \in [t_1, t_2]$.
\end{lemma}

\begin{proof}
First off, it should be noted that a solution to \ref{eq:theta-deriv} exists because the rhs is Lipschitz continuous. The proof of (1) follows directly from the fact that the derivative of $\theta(t)$ exist and is bounded. (2) is just a restatement of the intermediate value theorem.
\end{proof}

\begin{lemma}
  For a particle, $p$, with nonzero speed, and no centripetal acceleration for $t \ge t_0$, then
  \[
    \begin{cases}
      \theta(t) \to \pi \quad \text{as} \quad t \to \infty \qquad &\text{if} \;\; \theta(t_0) > 0\\
      \theta(t) = 0 \quad \text{for} \quad t \ge t_0 \qquad &\text{if} \;\; \theta(t_0) = 0
    \end{cases}
  \]
\end{lemma}

\begin{proof}
The second case is the easiest to check, 


\end{proof}