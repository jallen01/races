\begin{definition}
A n-dimensional path $\gamma(t): \mathrm{R} \to \mathrm{R}^n$ is a function which maps a time $t \in \R$, $0 \ge t \ge T_{f, \gamma}$, to a position $\bvec{X} \in \mathrm{R}^n$. 
\end{definition}

\begin{definition}
  A n-dimensional particle, $p$, is an object with zero volume that travels along a n-dimensional path. The particle may have conditions on its position, velocity, and acceleration in $\R^n$.
\end{definition}

\begin{definition}
  A valid path $\gamma(t)$ for a particle $p$ is a path such that all conditions on the particle are satisfied at every point along the path.
\end{definition}

\begin{definition}
  A path between two points, $\bvec{X_1}$ and $\bvec{X_2}$ is a path, $\gamma(t)$ where $\gamma(0) = \bvec{X_1}$ and $\gamma(T_{f, \gamma}) = \bvec{X_2}$.
\end{definition}

\begin{definition}
  For a given particle, $p$, a fastest path, $\hat{\gamma}(t)$, between two points, $\bvec{X_1}$ and $\bvec{X_2}$, is a valid path such that $T_f(\hat{\gamma}) \leq T_f(\gamma)$ for all valid paths, $\gamma(t)$, between $\bvec{X_1}$ and $\bvec{X_2}$.
\end{definition}

\begin{definition}
  The centripetal acceleration, $\bvec{a_c}$, of a particle, $p$, is the component of the acceleration of $p$ perpendicular to its direction of motion, $\bhat{v}$. The sign of $a_c$ is difined as the sign of the projection of $\bhat{a_c}$ onto $\bhat{r}$.
\end{definition}

\begin{definition}
  The tangential acceleration, $\bvec{a_t}$, of a particle, $p$, is the component of the acceleration of $p$ in its direction of motion, $\bhat{v}$.
\end{definition}