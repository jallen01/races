
\documentclass[12pt]{amsart}   % LaTeX with AMS style; 12 point for old eyes

\usepackage{amsmath,amssymb,amsfonts}   % For better support of math
\usepackage{graphicx}	        % Enable for eps and pdf figures, if they occur
\usepackage{caption}
\usepackage{float}
\usepackage{subcaption}
\usepackage{epstopdf}
\usepackage{hyperref} % Enable embedded hyperlinks.
\hypersetup{
	hidelinks, colorlinks, linkcolor=black, citecolor=black, urlcolor=red
}
        
% Commands to force sequential numbering:

\newtheorem{theorem}{Theorem}[section]
\newtheorem{proposition}[theorem]{Proposition}
\newtheorem{lemma}[theorem]{Lemma}
\newtheorem{definition}[theorem]{Definition}
\newtheorem{examples}[theorem]{Examples}
\newtheorem{remarks}[theorem]{Remarks}
\newtheorem{corollary}[theorem]{Corollary}
\newtheorem{remark}[theorem]{Remark}
\newtheorem{example}[theorem]{Example}
\newtheorem{conjecture}[theorem]{Conjecture}

% Define abs norm, paren, bracket, cbracket, and innerproduct
\usepackage{mathtools}
\DeclarePairedDelimiter\tempabs{\lvert}{\rvert}
\DeclarePairedDelimiter\tempnorm{\lVert}{\rVert}
\DeclarePairedDelimiter\tempinnerproduct{\langle}{\rangle}
\DeclarePairedDelimiter\tempparen{(}{)}
\DeclarePairedDelimiter\tempbracket{[}{]}
\DeclarePairedDelimiter\tempcbracket{\{}{\}}
\DeclareMathOperator*{\argmin}{arg\,min}
\DeclareMathOperator*{\argmax}{arg\,max}
% Swap * functionality
\makeatletter
\def\abs{\@ifstar{\tempabs}{\tempabs*}}
\def\norm{\@ifstar{\tempnorm}{\tempnorm*}}
\def\innerproduct{\@ifstar{\tempinnerproduct}{\tempinnerproduct*}}
\def\paren{\@ifstar{\tempparen}{\tempparen*}}
\def\bracket{\@ifstar{\tempbracket}{\tempbracket*}}
\def\cbracket{\@ifstar{\tempcbracket}{\tempcbracket*}}
\makeatother

% blackboard bold , for ``complex,'' etc
\newcommand{\C}{\mathbb C} 
\newcommand{\R}{\mathbb R} 
\newcommand{\Z}{\mathbb Z}
\newcommand{\Q}{\mathbb Q}
\newcommand{\N}{\mathbb N}

% Add fig with caption command. Inputs: {path}{caption}{label}
\newcommand{\includefig}[3]{
  \begin{figure}[H]
    \begin{center}
      \includegraphics[width=4in,height=3in,keepaspectratio]{#1}
    \end{center}
  \vspace{-.2in} % corrects bad spacing
  \caption{#2.\label{fig:#3}}
  \end{figure}
}

\newcommand{\bvec}[1]{\mathbf{#1}}
\newcommand{\bhat}[1]{\mathbf{\hat{#1}}}

\begin{document}

\graphicspath{ {figures/} }

\title[Optimal Paths in R2]{Optimal Particle Paths Around Points in $\R^2$ with Constrained Acceleration} 
 
\author{Jonathan Allen, John Wang}
\date{\today}

\maketitle

\subsection*{Abstract}

Optimal particle paths in $\R^2$ between points and around points are investigated in this paper. A paramerization of a particle's motion around a point is defined. From this parameterization, we derive a general solution for an optimal path with constant speed between 2 points. The results of this paper can by applied towards calculating optimal object trajectories in physics when acceleration is constrained. 

\section{Introduction}
The problem of finding optimal paths between points for a particle with fixed speed is interesting because many attributes of an optimal path are intuitively obvious, but rigorously proving them is much more difficult. One's first assumption would be that the optimal path between points A and B, would be to travel in a straight line from point A to B. This would be true in the simple case where the turning rate is not constrained, but it is not true in the general case when a particle has a constrained turning rate and starts with some nonzero velocity not directed towards B. 

\subsection{The Problem}

Given a particle that starts at a position in $\R^2$, has initial velocity, and has constrained acceleration, what is the path in $\R^2$ for the particle to take to reach a final position in $\R^2$, such that the time spent traveling from the starting position to the ending position is minimized. The particle can have restrictions on its position, velocity, and acceleration.

\subsection{Motivation}

The problem of finding a fastest path for a particle with constant speed is usefull because it is identical to the problem of finding a shortest path in $\R^2$ given constraints on the radius of curvature of the path, which appears in fields such as physics and computational geometry. One could imagine a race car driver navigating a racetrack with constant speed, as a physical example of the problem.

This example directly motivates our problem, because a racecar driver has a turning rate limited by friction.

Understanding how to find optimal paths for a particle with constant speed moving in $\R^2$ will provide insight into solving even more complicated problems with varying speed and more dimensions such as spacecraft trajectories.

 

\subsection{Outline}

We first present a modified polar coordinate representation of the particle's position around a point and derive a set of differential equations governing the motion of the particle in this coordinate system. This allows us to devise some simple lemmas about the motion of the particle.

\section{Notation}
\subsection{Vectors}

\begin{description}
  \item[$a$] Scalar quantity.
  \item[$\bvec{a}$] Vector in n-dimensional space. $\norm{\bvec{a}} = a$.
  \item[$\bhat{a}$] Unit vector in n-dimensional space. $\bhat{a} = \bvec{a}/a$ and $\norm{\bhat{a}} = 1$.
\end{description}

\subsection{Angles}

Since many of the theorems in this paper involve polar coordinates, we define two different types of angle measurements: the standard measurement, and a directional measurement. All angles are measured in radians.

In the standard measurement, angles are in $[0, 2\pi]$ and are measured counterclockwise. In the directional measurement, angles are in $[-\pi, \pi]$ and the measurement direction is indicated by an arrow on the figure.

\begin{figure}[H]
  \centering
  \begin{subfigure}[b]{0.4\textwidth}
    \includegraphics[width=\textwidth]{angle_def_1.eps}
    \caption{}
    \label{fig:angle-def-1}
  \end{subfigure}
  \qquad \qquad
  \begin{subfigure}[b]{0.4\textwidth}
    \includegraphics[width=\textwidth]{angle_def_2.eps}
    \caption{}
    \label{fig:angle-def-2}
  \end{subfigure}
  \caption{Standard angle notation (no arrow)}
\end{figure}

\begin{figure}[H]
  \begin{subfigure}[b]{0.4\textwidth}
    \includegraphics[width=\textwidth]{angle_def_3.eps}
    \caption{}
    \label{fig:angle-def-3}
  \end{subfigure}
  \qquad \qquad
  \begin{subfigure}[b]{0.4\textwidth}
    \includegraphics[width=\textwidth]{angle_def_4.eps}
    \caption{}
    \label{fig:angle-def-4}
  \end{subfigure}
  \caption{Directional angle notation (arrow)}
\end{figure}

\section{Particles and Paths}
\begin{definition}
A n-dimensional path $\gamma(t): \mathrm{R} \to \mathrm{R}^n$ is a function which maps a time $t \in \R$, $0 \ge t \ge T_{f, \gamma}$, to a position $\bvec{X} \in \mathrm{R}^n$. 
\end{definition}

\begin{definition}
  A n-dimensional particle, $p$, is an object with zero volume that travels along a n-dimensional path. The particle may have conditions on its position, velocity, and acceleration in $\R^n$.
\end{definition}

\begin{definition}
  A valid path $\gamma(t)$ for a particle $p$ is a path such that all conditions on the particle are satisfied at every point along the path.
\end{definition}

\begin{definition}
  A path between two points, $\bvec{X_1}$ and $\bvec{X_2}$ is a path, $\gamma(t)$ where $\gamma(0) = \bvec{X_1}$ and $\gamma(T_{f, \gamma}) = \bvec{X_2}$.
\end{definition}

\begin{definition}
  For a given particle, $p$, a fastest path, $\hat{\gamma}(t)$, between two points, $\bvec{X_1}$ and $\bvec{X_2}$, is a valid path such that $T_f(\hat{\gamma}) \leq T_f(\gamma)$ for all valid paths, $\gamma(t)$, between $\bvec{X_1}$ and $\bvec{X_2}$.
\end{definition}

\begin{definition}
  The centripetal acceleration, $\bvec{a_c}$, of a particle, $p$, is the component of the acceleration of $p$ perpendicular to its direction of motion, $\bhat{v}$. The sign of $a_c$ is difined as the sign of the projection of $\bhat{a_c}$ onto $\bhat{r}$.
\end{definition}

\begin{definition}
  The tangential acceleration, $\bvec{a_t}$, of a particle, $p$, is the component of the acceleration of $p$ in its direction of motion, $\bhat{v}$.
\end{definition}

\section{Traveling Between Points}
Now that we have defined our notation, we are ready to think about the problem of optimal trajectories.

The simplest problem to tackle first, is the optimal trajectory from a starting position, $\bvec{X_0}$ to a final position, $\bvec{X_f}$. Working in a polar coordinate system centered on $X_f$, it is clear, that our goal is achieved when $r = 0$. Furthermore, $\theta(t)$ must be $0$ before reaching $X_f$, otherwise, it will grow exponentially fast as $r(t) \to 0$, which can be seen in equation (\ref{eq:r-deriv}).

\begin{lemma}
If $a_t=0$ and $a_c \le a_{c,max}$, then the minimum radius of curvature of a particle's trajectory is given by
\begin{equation}
R_{min} = \frac{v^2}{a_{c,max}}
\end{equation}
\end{lemma}

\begin{proof}
This comes directly from the definition of radius of curvature 

\[
R = \frac{v^2}{a_c}
\]

and the fact that 

\[
a_c \le a_{c,max} \to \frac{1}{a_c} \ge \frac{1}{a_{c,max}}
\]
\end{proof}

\begin{theorem}
A 2-dimensional particle, $p$, traveling along a valid path, $\hat{\gamma(t)}$, from $X_1$ to $X_2$ has $\frac{d\abs{\theta(t)}}{dt} < 0$ for all $t \in [0, T_{f, \gamma}]$, unless $\norm{X_1 - X_2} < 2 R_{min}$.
\end{theorem}

\begin{proof}
Since $\theta(T_{f,\gamma}) = 0$, then if $\frac{d\abs{\theta(t)}}{dt} > 0$ for $t \in [t_1, t_2]$, since $\theta{T_{f,\gamma}} < \theta{t_1}$, then by Lemma \ref{lemma:intermediate-theta} $\theta(t_1) = \theta(t_2)$ for some $t_2 \in [t_1, T_{f, \gamma}$. Since all of the time derivatives in (\ref{eq:r-deriv}) - (\ref{eq:theta-deriv}) are all monotonic for $r \in (0, \infty)$, there is no local maximum for them, so the optimal strategy should be the same, regardless of $r$. In other words, there can not be multiple cases for optimal strategies based on $r$.
\end{proof}

\begin{theorem}
Given a 2-dimensional particle, $p$, it is always optimal to minimize $\frac{d\abs{\theta(t)}}{dt}$ at all $t \in [0, T_{\gamma}]$.
\end{theorem}

\begin{proof}
[TODO]
\end{proof}

\begin{theorem}
Given a 2-dimensional particle, $p$, with bounded centripetal acceleration, $a_c \le a_{c,max}$ and zero tangential acceleration, every valid optimal path, $\hat{\gamma}(t)$ between two points $X_1$ and $X_2$, has a radius of curvature of $0$ or $a_{c,max}$ at every point along it.
\end{theorem}

\begin{proof}
[TODO]
\end{proof}

\begin{corollary}\label{corr:optimal-path-fix-s}
Given a 2-dimensional particle, $p$, with bounded centripetal acceleration, $a_c \le a_{c,max}$ and zero tangential acceleration, every valid optimal trajectory is constructed from line segments and circular sections of radius $R_{min}$, that are tangent at their intersections.
\end{corollary}

\begin{proof}
[TODO]
\end{proof}

% -----------------------------------------------------------------------------

Now we will show that the fastest path between two points goes in a straight line. This will formalize the natural intuition that straight lines are the fastest path when there are no obstacles between the start and finish position.

\begin{theorem}
  Given a 2-dimensional particle $p$ whose initial position is $\bvec{X_1}$ and moves with bounded tangential acceleration, $a_t \le a_{t,max}$, and infinite centripetal acceleration, the fastest path $\hat{\gamma}(t)$ which $p$ can trace from $\bvec{X_1}$ to $\bvec{X_2}$ lies on the line segment from $\bvec{X_1}$ to $\bvec{X_2}$:
\end{theorem}

\begin{proof} 

Without loss of generality, we can define a cartesian coordinate system, where the origin is at $\bvec{X_1}$ and the positive x-axis passes through $\bvec{X_2}$.

Now let us examine the particle's motion in the $\bhat{x}$ direction. The speed of the particle is the following
\begin{equation}
  v_x(t) = \int_0^t a_x(t_1) dt_1
\end{equation}

To find the distance $l_x(t)$ travelled in the $\bhat{x}$ direction, we can use the relation:
\begin{align}
  d(t) &= \int_0^t v_x(t_2) dt_2 \\
  &= \int_0^t \int_0^t a_x(t_1) dt_1 dt_2
\end{align}

Recall that the tangential acceleration of the point mass $p$ is bounded by $a_{t,max}$. This means that $a_t(t) \leq \bar{a}$ for all $t$. Therefore, we see:
\begin{align}
  d(t) &\leq \int_0^t \int_0^t \bar{a} dt_1 dt_2\\
  &= \frac{\bar{a} t^2}{2}
\end{align}

Thus, in order to travel a distance of $d(T_f) = x_2$, it needs to be the case that $T_f \geq \sqrt{\frac{2 x_2}{\bar{a}}}$. Moreover, equality holds if and only if $a_t(t) = \bar{a}$ for all $t \in [0, T_f(\gamma)]$.

If the point mass travels for time $t < \sqrt{\frac{2 x_2}{\bar{a}}}$, then it is impossible for the point mass to reach $(x_2, 0)$ when starting at $(0,0)$. This is because $p$ cannot reach $(x_2, 0)$ in the $x$ direction when $t < \sqrt{\frac{2 x_2}{\bar{a}}}$ and any acceleration in the $y$ direction would not enable this either.

This means that the fastest path is completed in time $T_f(\hat{\gamma}) = \sqrt{\frac{2 x_2}{\bar{a}}}$. Let us examine the path taken by the point mass $p$ on this fastest path. Recall that $a_t(t) = \bar{a}$ for all $t$ along the fastest path. This means that there was no centripetal acceleration $|a_c| = 0$. In other words, the point mass never turned on its way to reaching the destination point. The only way this could have happened is if it travelled along the $x$ axis in a straight line.

Now, we have seen that the fastest path in the transformed coordinates travels exactly on the $x$ axis so that $y = 0$ anywhere along the fastest path. Notice, however, that the $x$ axis in the transformed coordinates is given exactly by the following line:
\begin{equation}
  y = \frac{y_2 - y_1}{x_2 - x_1} x + y_1
\end{equation}

Thus, we see that the fastest path in the original coordinates follows the above equation, which is what we wanted to show.
\end{proof}

\begin{corollary}
  Given points $(x_1, y_1), (x_2, y_2) \in \mathrm{R}^2$ and a particle $p$ whose initial position is $(x_1, y_1)$ which moves with acceleration bounded by $\bar{a}$, the fastest path $\hat{\gamma}(t)$ which $p$ can trace from $(x_1, y_1)$ to $(x_2, y_2)$ is unique.
\end{corollary}
\proof We have already shown that any fastest path between $(x_1, y_1)$ and $(x_2, y_2)$ follows the straight line given by $y = \frac{y_2 - y_1}{x_2 - x_1} x + y_1$. Moreover, we showed that when travelling along the fastest path, the particle must have acceleration along the straight line of $\bar{a}$. Since we have starting position $(x_1, y_1)$ and initial speed of $0$, the acceleration of the particle $a(t)$ uniquely defines a path for the particle.

There is only a single function $a(t) = \bar{a}$ which the acceleration can satisfy when the particle is moving along a fastest path, therefore, there is only a single possible fastest path.
\qed

% -----------------------------------------------------------------------------



\section{Conclusion}
In this paper, we have formalized many of the things that intuition would tell us. Namely, we have shown that a straight line is the fastest way to get between two points. This fact is unsurprising because of the fact that acceleration in a single direction (toward the finish point) is the least wasteful means of getting to the finish.

We developed a number of simple lemmas governing the motion of a volume-less particle with bounded acceleration. We also showed that symmetric paths tend to be better than non-symmetric paths (through our simple quadrilateral path minimization problem).

Finally, we found an optimal path for a particle to trace around three cones when the particle has constant velocity.


\section*{Appendix}
\subsection{Coordinate Systems} 

\subsubsection{Radius of Curvature}

The radius of curvature, $R$, of a curve at a point is a measure of the radius of the circular arc which best approximates the curve at that point.

For $a_t = 0$

\[
R= \abs{\frac{v^2}{a_c}}
\]

\subsection{Vector Calculus in Polar Coordinates}

\begin{align*}
\boldsymbol{x}& = r \boldsymbol{\hat{r}}\\
\boldsymbol{\vec{v}}& = \dot{r} \boldsymbol{\hat{r}} + r \dot{\phi} \boldsymbol{\hat{\phi}}\\
\label{eq:a_vec_def}\boldsymbol{\vec{a}}& = \paren{\ddot{r} - r \dot{\phi}^2} \boldsymbol{\hat{r}} + \frac{1}{r} \frac{d}{dt} \paren{r^2 \dot{\phi}} \boldsymbol{\hat{\phi}}
\end{align*}

\bibliographystyle{plain}
\begin{thebibliography}{9}
\bibitem{wiki_pcoords}
\url{http://en.wikipedia.org/wiki/Polar_coordinate_system}
\end{thebibliography}

\end{document}
