
\documentclass[12pt]{amsart}   % LaTeX with AMS style; 12 point for old eyes

\usepackage{amsmath,amssymb,amsfonts}   % For better support of math
\usepackage{graphicx}	        % Enable for eps and pdf figures, if they occur
\usepackage{hyperref} % Enable embedded hyperlinks.
\hypersetup{
	hidelinks, colorlinks, linkcolor=black, citecolor=black, urlcolor=red
}
        
% Commands to force sequential numbering:

\newtheorem{theorem}{Theorem}[section]
\newtheorem{proposition}[theorem]{Proposition}
\newtheorem{lemma}[theorem]{Lemma}
\newtheorem{definition}[theorem]{Definition}
\newtheorem{examples}[theorem]{Examples}
\newtheorem{remarks}[theorem]{Remarks}
\newtheorem{corollary}[theorem]{Corollary}
\newtheorem{remark}[theorem]{Remark}
\newtheorem{example}[theorem]{Example}
\newtheorem{conjecture}[theorem]{Conjecture}

% Define abs norm, paren, bracket, cbracket, and innerproduct
\usepackage{mathtools}
\DeclarePairedDelimiter\tempabs{\lvert}{\rvert}
\DeclarePairedDelimiter\tempnorm{\lVert}{\rVert}
\DeclarePairedDelimiter\tempinnerproduct{\langle}{\rangle}
\DeclarePairedDelimiter\tempparen{(}{)}
\DeclarePairedDelimiter\tempbracket{[}{]}
\DeclarePairedDelimiter\tempcbracket{\{}{\}}

% Swap * functionality
\makeatletter
\def\abs{\@ifstar{\tempabs}{\tempabs*}}
\def\norm{\@ifstar{\tempnorm}{\tempnorm*}}
\def\innerproduct{\@ifstar{\tempinnerproduct}{\tempinnerproduct*}}
\def\paren{\@ifstar{\tempparen}{\tempparen*}}
\def\bracket{\@ifstar{\tempbracket}{\tempbracket*}}
\def\cbracket{\@ifstar{\tempcbracket}{\tempcbracket*}}
\makeatother

\begin{document}

\title[The 18.821 report]{The 18.821 Mathematics Project Lab Report 
[Proofs]} 
% the first [...] gives a brief version of the title, which is long!
 
\author{Jonathan Allen}
\date{\today}              % or an actual date


\newcommand{\C}{\mathbb C} % blackboard bold , for ``complex,'' etc
\newcommand{\R}{\mathbb R} 
\newcommand{\Z}{\mathbb Z}
\newcommand{\Q}{\mathbb Q}
\newcommand{\N}{\mathbb N}

\maketitle

\section{Theorems}

\subsection{Notation\label{sec:notation}} 

\begin{description}
    \item[$t$] Time
    \item[$l$] Path length
    \item[$s$] Speed
    \item[$a_t$] Tangential acceleration
    \item[$a_c$] Centripetal acceleration
    \item[$\vec{x}$] Position
    \item[$\vec{v}$] Velocity
    \item[$\vec{a}$] Acceleration
\end{description}

\section{Coordinates}

\subsection{Scalar Calculus}

\begin{align}
s& = \frac{dl}{dt}\\
%
a_t& = \frac{ds}{dt}\\
\end{align}

\subsection{Vector Calculus}

\begin{align}
\boldsymbol{x}& = r \boldsymbol{\hat{r}}\\
%
\boldsymbol{\vec{v}}& = \dot{r} \boldsymbol{\hat{r}} + r \dot{\phi} \boldsymbol{\hat{\phi}}\\
%
\boldsymbol{\vec{a}}& = \paren{\ddot{r} - r \dot{\phi}^2} \boldsymbol{\hat{r}} + \frac{1}{r} \frac{d}{dt} \paren{r^2 \dot{\phi}} \boldsymbol{\hat{\phi}}
\end{align}

\subsection{Relations}

\begin{align}
l& = \int_{t=0}^{T} \norm{\boldsymbol{\vec{v}} \, } \; dt\\
%
s& = \boldsymbol{\vec{v}}\\
%
a_t& = \norm{\boldsymbol{\vec{a}}\,} \frac{\boldsymbol{\vec{v}}}{\norm{\boldsymbol{\vec{v}}\,}}
\end{align}

\section{Traveling Between Points}

\begin{definition}
  A path $\gamma: \mathrm{R} \to \mathrm{R}^2$ is a function which maps some time $t$ to a position $\gamma(t) \in \mathrm{R}^2$. The path is defined from time $t = 0$ until the end time of the path, denoted as $T_{f}(\gamma)$.
\end{definition}

\begin{definition}
  A valid path $\gamma: \mathrm{R} \to \mathrm{R}^2$ for some point mass $p$ and conditions $\mathrm{C}$ is some path which at all times $t$ such that $0 \leq t \leq T_f(\gamma)$, all conditions in $\mathrm{C}$ are satisfied.
\end{definition}

\begin{definition}
  A valid targetted path $\gamma: \mathrm{R} \to \mathrm{R}^2$ for some point mass $p$, conditions $\mathrm{C}$, starting point $\vec{x_1}$, and ending point $\vec{x_2}$ is a valid path where $\gamma(t) = \vec{x_1}$ and $\gamma(T_f(\gamma)) = \vec{x_2}$. In other words, it is a valid path which starts at $\vec{x_1}$ and ends at $\vec{x_2}$.
\end{definition}

\begin{definition}
  A fastest path $\hat{\gamma}: \mathrm{R} \to \mathrm{R}^2$ for a particular point mass $p$, a starting point $\vec{x_1}$, a destination point $\vec{x_2}$, and some set of conditions $\mathrm{C}$ is a valid targetted path $\hat{\gamma}$ such that $T_f(\hat{\gamma}) \leq T_f(\gamma)$ for all valid targetted paths $\gamma$ with the same $p$, $\vec{x_1}$, $\vec{x_2}$, and $\mathrm{C}$.
\end{definition}

\begin{theorem}
  Given points $(x_1, y_1), (x_2, y_2) \in \mathrm{R}^2$ and a point mass $p$ whose initial position is $(x_1, y_1)$ which moves with tangential acceleration bounded by $\bar{a}$, the fastest path $\hat{\gamma}(t)$ which $p$ can trace from $(x_1, y_1)$ to $(x_2, y_2)$ follows the straight line where all coordinates $(x,y)$ on the straight line are given by:
  \begin{eqnarray}
    y = \frac{y_2 - y_1}{x_2 - x_1} x + y_1
  \end{eqnarray}
\end{theorem}
\proof Let's transform the problem. We can reset our coordinate axes so that $(x_1, y_1)$ is set to the origin and $(x_2, y_2)$ is on the x-axis. In this new coordinate system, we have transformed the following:
\begin{eqnarray}
  (x_1, y_1) &\to& (0,0) \\
  (x_2, y_2) &\to& (x_2', 0)
\end{eqnarray}

For convenience of notation, we will now refer to $x_2'$ as $x_2$.
\qed

\begin{corollary}
  The fastest path between two points $(x_1, y_1), (x_2, y_2) \in \mathrm{R}^2$ is unique.
\end{corollary}

\bibliographystyle{plain}
\begin{thebibliography}{9}
\bibitem{wiki_pcoords}
\url{http://en.wikipedia.org/wiki/Polar_coordinate_system}
\end{thebibliography}

\end{document}

From Spring 2011 or before; revised by Miller, Spring 2013. 
