\subsection{Quadrilateral}

Before we attempt to tackle the problem of finding the optimal path around a cone at constant velocity, we may want to find some intuition for the problem with a simple, but related problem. In particular, if a particle is traveling around a cone, we would like to know what circle it traces as it goes around the cone.

To make a simple model of this problem, we will use a quadrilateral to model a path around a cone. We will construct a quadrilateral with particular constaints that models the constraints of a path around a cone, and we will try to find the parameters that minimize the perimeter around the quadrilateral. Let us call the sides of the quadrilateral $s_1, s_2, s_3$ and $s_4$. Imagine that $s_1$ and $s_2$ are on opposite sides of the quadrilateral with fixed lengths $l_1$ and $l_2$ respectively. Now we will constrain the problem to be related to the problem of a particle traveling around a cone: imagine $s_1$ and $s_2$ are connected by a bar $s_b$ of length $d$. The bar will be perpendicular to $s_1$. The angle that $s_b$ forms with $s_2$ will be called $\theta$. We will try to find the optimal $\theta$ that minimizes the perimeter around the quadrilateral.

We will use figure \label{fig:quadrilateral} as reference.

\includefig{quad.eps}{Quadrilateral path. A $\to$ B $\to$ C $\to$ D}{quad}

\subsection{Symmetric Quadrilateral}

Let us begin by solving the simplest version of this problem. Let us imagine that $s_b$ is connected to the midpoints of $s_1$ and $s_2$. We know that the perimeter will be given by $l_1 + l_2 + l_3 + l_4$ where $l_3$ and $l_4$ are the lengths of the sides of $s_3$ and $s_4$, respectively. We are given $l_1$ and $l_2$, but we will need to compute $l_3$ and $l_4$ as functions of $l_1, l_2,d, $ and $\theta$. 

We can find $l_3$ by using the fact that it forms a right triangle. We know that $l_3^2 = m^2 + n^2$. Finding $l_4$ is similar. Therefore, we just need to find $m$ and $n$. This can be done by using the fact that $m = \frac{l_1}{2} - \frac{l_2}{2} \sin \theta$. This is just the upper half of $s_1$ minus the projection of $s_2$ onto $s_1$. We can find $n$ similarly: $n = d - \frac{l_2}{2} \cos \theta$. This is just the bar $s_b$ minus the projection of $s_2$ onto the bar. Thus by substituting, we have:
\begin{eqnarray}
  l_3 = \sqrt{ \left(\frac{l_1}{2} - \frac{l_2}{2} \sin \theta \right)^2 + \left( d - \frac{l_2}{2} \cos \theta \right)^2 }
\end{eqnarray}

We can do a similar analysis on $l_4$, only remembering that $n$ for $l_4$ gets extended by the projection onto $s_4$ instead of shrunken. We therefore have:
\begin{eqnarray}
  l_4 = \sqrt{ \left(\frac{l_1}{2} - \frac{l_2}{2} \sin \theta \right)^2 + \left( d + \frac{l_2}{2} \cos \theta \right)^2 }
\end{eqnarray}

Now, to minimize the perimeter with respect to $\theta$, we want to minimize $l_1 + l_2 + l_3 + l_4$. Since we know that $l_1$ and $l_2$ are fixed, we really want to minimize $l_3 + l_4$ with respect to $\theta$. The other thing to note is that we're minimizing positive distances. We will invoke the following lemma so that we can simplify our expression for $\min l_3 + l_4$:

\begin{lemma}
  If $f(t), g(t) > 0$ and $k(t) > 0$ is a strictly monotonically increasing function for all $t$, then $\argmin_t k(f(t)) + k(g(t)) = \argmin_t f(t) + g(t)$.
  \label{lemma:minsqrt}
\end{lemma}
\proof Let $t_1, t_2$ be such that $f(t_1) + g(t_1) < f(t_2) + g(t_2)$. In this proof, we will show that $k(f(t_1)) + k(g(t_1)) < \sqrt{f(t_2)} + \sqrt{g(t_2)}$. Since $k(t)$ is a strictly monotonically increasing function when $t > 0$, we know that $k(x) < k(y)$ if and only if $x < y$ (assuming we can confine $x, y$ to be non-negative).

Because this is the case, we see that $k(f(x)) < k(f(y))$ if and only if $f(x) < f(y)$ (the same goes for $g$). Thus, we see that if we have found the minimum $t_m$ to $\argmin_t k(f(t)) + k(g(t))$, then it is the case that $k(f(t_m)) + k(g(t_m)) < k(f(t)) + k(g(t))$ for all $t \neq t_m$ (again where $t > 0$). Following our train of logic, we see that $f(t_m) + g(t_m) < f(t) + g(t)$ for all $t \neq t_m$, which means that $t_m$ is a minimum of $f(t) + g(t)$. Thus by finding a minimum $t_m$ to $k(f(t)) + k(g(t))$, we also found a minimum to $f(t) + g(t)$. \qed

Since we've proven this lemma, we can invoke it upon $\min l_3 + l_4$. Since $l_3 = \sqrt{z_3}$ and $l_4 = \sqrt{z_4}$, we can use $k(t) = \sqrt{t}$ and we can write $\min l_3 + l_4 = \min z_3 + z_4$ by using our lemma. Thus, we now want to solve the problem:
\begin{eqnarray}
  \argmin_{\theta} &2& \left(\frac{l_1}{2} - \frac{l_2}{2} \sin \theta \right)^2 \\
  &+& \left( d - \frac{l_2}{2} \cos \theta \right)^2 + \left( d + \frac{l_2}{2} \cos \theta \right)^2
\end{eqnarray}

Now, we can expand out our expression and use the fact that $\sin^2 \theta + \cos^2 \theta = 1$ to obtain a much simpler (but equivalent) minimization problem:
\begin{eqnarray}
  \argmin_{\theta} \frac{l_1^2}{2} + \frac{l_2^2}{2} + 2d^2 - l_1 l_2 \sin \theta
\end{eqnarray}

We note that $l_1, l_2,$ and $d$ are all constants which are given to us in the problem. Therefore, the minimization problem really boils down to
\begin{eqnarray}
  \argmin_{\theta} - l_1 l_2 \sin \theta = \argmax_{\theta} \sin \theta
\end{eqnarray}

% -----------------------------------------------------------------------------

% 3 cones, constant speed
\begin{theorem}

Given a cone setup consisting of 3 cones at locations $X_A$, $X_B$, and $X_C$.

\includefig{cones_fix_s.eps}{A}{cones-fix-s}

\begin{align}
dB& = \sqrt{x_B + y_B}\\
\intertext{Applying the triangle inequality...}
d_c& = \alpha_c - \tan^{-1}\paren{\frac{y_B}{x_B}}\\
\end{align}

\includefig{cones_fix_s_2.eps}{A}{cones-fix-s-2}

\begin{align}
l_1& = \sqrt{d_c - r}\\
x_c = x_B - r \cos{\alpha_c}\\
y_c = y_B - r \sin{\alpha_c}\\
d_c& = \alpha_c - \tan^{-1}\paren{\frac{y_c}{x_c}}\\
\end{align}

\end{theorem}

\proof