Now we are finally able to tackle the model problem of optimal paths around cones. First we will solve a simpler but related problem of minimizing the path length around a quadrilateral.

\subsection{Quadrilateral}

We would like to minimize the path length around the quadrilateral shown in Figure \ref{fig:quad}.

\includefig{quad.eps}{Quadrilateral path. A $\to$ B $\to$ C $\to$ D}{quad}

We will constrain the problem such that it is similar to the problem of a particle traveling around a cone. Sides $\overline{BC}$, $\overline{DA}$, and $\overline{EF}$ have fixed lengths $l_1$, $l_2$, and $w$ respectively. Points E and F bisect $\overline{DA}$ and $\overline{BC}$ respectively. The goal is

\[
\argmin_{\theta} \bracket{l_{AB} + l_{BC} + l_{CD}}
\]

\begin{align}
  x_c& = w - \frac{l_2}{2 \cos{\theta}}\\
  y_c& = \frac{l_2}{2} - \frac{l_2}{2 \sin{\theta}}\\
  l_{AB}& = \sqrt{x_c^2 + y_c^2}\\
  & = \sqrt{ \paren{\frac{l_1}{2} - \frac{l_2}{2 \cos\theta}}^2 + \paren{w - \frac{l_2}{2 \sin\theta}}^2 }\\
  l_{BC}& = l_2\\
  l_{CD}& = \sqrt{\paren{\frac{l_1}{2} + \frac{l_2}{2 \cos\theta}}^2 + \paren{w - \frac{l_2}{2 \sin\theta}}^2 }
\end{align}

Since we know that $l_{BC}$ is fixed the problem reduces to

\[
\argmin_{\theta} \bracket{l_{AB} + l_{BC} + l_{CD}}
\]

Because all distances are positive, we can use Lemma \ref{lemma:conv-argmin}, defined in the Appendix, to simplify our problem:

If we define $l_{AB} = \sqrt{z_{AB}}$ and $l_{CD} = \sqrt{z_{CD}}$, we can use \ref{lemma:conv-argmin} with $k(t) = \sqrt{t}$ to simplify our problem to

\begin{multline}
  \argmin_{\theta} \left[ 2\paren{\frac{l_1}{2} - \frac{l_2}{2 \sin\theta}}^2 \right.\\
  \left. + \paren{w - \frac{l_2}{2 \cos\theta}}^2 + \paren{w + \frac{l_2}{2 \cos\theta}}^2 \right]
\end{multline}

Now, we can expand out our expression and use Lemma the fact that $\sin^2 \theta + \cos^2 \theta = 1$ to obtain a much simpler (but equivalent) minimization problem:
\begin{eqnarray}
  \argmin_{\theta} \frac{l_1^2}{2} + \frac{l_2^2}{2} + 2d^2 - l_1 l_2 \sin \theta
\end{eqnarray}

We note that $l_1, l_2,$ and $d$ are all constants which are given to us in the problem. Therefore, the minimization problem really boils down to
\begin{eqnarray}
  \argmin_{\theta} - l_1 l_2 \sin \theta = \argmax_{\theta} \sin \theta
\end{eqnarray}

% -----------------------------------------------------------------------------

% 3 cones, constant speed
\begin{theorem}

Given a particle with zero tangential acceleration and bounded centripetal acceleration, $a_c \le a_{c,max}$, and a cone setup consisting of 3 cones, A, B, and C, at locations $\bvec{X_A}$, $\bvec{X_B}$, and $\bvec{X_C}$ as is shown in Figure \ref{fig:3-cones}

\includefig{3_cones.eps}{3 cone setup}{3-cones}

the optimal path around the cones is shown in Figure \ref{fig:3-cones-path}

\includefig{3_cones_path.eps}{3 cone optimal path}{3-cones-path}

\end{theorem}

\begin{proof}
[TODO: Finish proof]

Since speed is constant, the optimal path will be the shortest path. By Corollary \ref{corr:optimal-path-fix-s}, the optimal path can only include line segments and circular sections. Thus, the problem reduces to a trigonometric problem.

The turn must touch cone B, otherwise the trajectory will either not go around the cone, or be longer than necessary. Therefor, the center of the turn, c, must be a distance $r$ away from cone B.

\includefig{cones_fix_s.eps}{Position of center of turn}{cones-fix-s}

Now that c is defined by $\alpha$, let us determine the total path length as a function of $\alpha$.

\includefig{cones_fix_s_2.eps}{Path length}{cones-fix-s-2}

\begin{align*}
x_c& = x_B - r \cos{\alpha_c}\\
y_c& = y_B - r \sin{\alpha_c}\\
l_1& = \sqrt{d_c - r}\\
l_2& = r \paren{\alpha_c - \tan^{-1}\paren{\frac{y_c}{x_c}}}\\
\end{align*}
\end{proof}