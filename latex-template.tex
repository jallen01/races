
\documentclass[12pt]{amsart}   % LaTeX with AMS style; 12 point for old eyes

\usepackage{amssymb,amsfonts}   % For better support of math
\usepackage{graphicx}	        % Enable for eps and pdf figures, if they occur
\usepackage{hyperref}           % Enable embedded hyperlinks.

% Commands to force sequential numbering:

\newtheorem{theorem}{Theorem}[section]
\newtheorem{proposition}[theorem]{Proposition}
\newtheorem{lemma}[theorem]{Lemma}
\newtheorem{definition}[theorem]{Definition}
\newtheorem{examples}[theorem]{Examples}
\newtheorem{remarks}[theorem]{Remarks}
\newtheorem{corollary}[theorem]{Corollary}
\newtheorem{remark}[theorem]{Remark}
\newtheorem{example}[theorem]{Example}
\newtheorem{conjecture}[theorem]{Conjecture}

\begin{document}

\title[The 18.821 report]{The 18.821 Mathematics Project Lab Report 
[Replace this with your own short descriptive title!]} 
% the first [...] gives a brief version of the title, which is long!
 
\author{Jonathan Allen, John Wang}
\date{\today}              % or an actual date


\newcommand{\C}{\mathbb C} % blackboard bold , for ``complex,'' etc
\newcommand{\R}{\mathbb R} 
\newcommand{\Z}{\mathbb Z}
\newcommand{\Q}{\mathbb Q}
\newcommand{\N}{\mathbb N}

\begin{abstract}
This is a \LaTeX\ template for 18.821, which you can use for your
own reports. 
\end{abstract}

\maketitle

\section{Introduction\label{sec:intro}}

%Your introduction section should convey the motivation and main point of the project. Avoid generalized philosophical or
%historical remarks (``Ever since the beginning of human culture \dots''), or lists of known results. Technical definitions are better kept out of the introduction. The aim should be to get the right
%informal picture into the reader's head. This may take some effort.

This brief document shows some examples of the use of \LaTeX\  
and indicates some special features of the Math Lab report style.
The \href{http://stellar.mit.edu/S/course/18/sp13/18.821/}{course website}
contains links to several \LaTeX\ manuals. 

End the introduction by describing the contents of the paper section by section, and which
team member(s) wrote each of them. For instance, Section \ref{sec:bibliography} discusses referencing, and is written by P. Gurps.



\section{\LaTeX\ Examples\label{sec:examples}}


Here are some ways of producing mathematical symbols. Some are pre-defined either in \LaTeX\ or in the AMS package which
this document loads. For instance, sums and integrals, $\sum_{i=1}^n 1 = n$, $\int_0^n x \, dx = n^2/2$.
We've defined a few other symbols at the start of the document, for instance $\N,\Q,\Z,\R$. You can make marginal notes for yourself or your co-authors like this:
\marginpar{Unfinished here?}

If you want to typeset equations, there are many choices, with or without numbering:
\[
\int_0^1 x\, dx = 1/2,
\]
or
$$
\sum_{i = 1}^\infty i = -\frac{1}{12}
$$
or
\begin{equation*}
1 - 1 + 1 - \cdots = \frac{1}{2}.
\end{equation*}
If you want a number for an equation, do it like this:
\begin{equation}\label{eq:first-equation}
\lim_{n \to \infty}\, \sum_{k = 1}^n \frac{1}{k^2} = \frac{\pi}{6}.
\end{equation} %
This can then be referred to as \eqref{eq:first-equation}, which is much easier than
keeping track of numbers by hand. To group several equations, aligning on the $=$ sign, do
it like this:
\begin{align*}
x_1 + 2x_2 + 3x_3 &= 7 \\ y &= mx + c \\ &= 4x - 9.
\end{align*}

You can easily embed hyperlinks into the output .pdf document: 
\href{http://stellar.mit.edu/S/course/18/sp13/18.821/}{click here} for example.

\section{Images}
Figure \ref{fig:2} is an example of a .pdf image put into a floating environment, which means LaTeX will draw it wherever there's enough space left in your manuscript.  %To draw images which can be saved as .pdf, use your favorite graphics program such as (under Linux or Unix) xfig or gimp.
%\begin{figure}
%\begin{center}
%\includegraphics[width=4in]{spiral.eps}
%\end{center}
%\vspace{-.2in} % corrects bad spacing
%\caption{My first .eps figure.\label{fig:2}}
%\end{figure}
Look at the .tex original to see how to insert a figure like this. 
%Alternatively, you can insert .pdf files using the same syntax:

\begin{figure}
\begin{center}
%\includegraphics[width=4in]{spiral.pdf}
\end{center}
\vspace*{-.5in} % corrects bad spacing
\caption{My first .pdf figure.\label{fig:2}}
\end{figure}

%An advantage of using .pdf figures is that you can produce .pdf files 
%from latex in one step, using the command {\tt pdflatex}, rather than
%the three-step process described below. 

\section{Theorems and such}

An example of a ``conjecture environment'' is given below, in Conjecture
\ref{conj:vaught}.  Theorems, lemmas, propositions, definitions, and
such all use the same command with the appropriate name changed.
In fact, if you look at the top of this .tex file, you can see where we've defined
these environments.

\begin{conjecture}[Vaught's Conjecture]\label{conj:vaught}
Let $T$ be a countable complete theory.  If $T$ has fewer than
$2^{\aleph_0}$ many countable models (up to isomorphism), then it has
countably many countable models.
\end{conjecture}

\begin{theorem}
When it rains it pours.
\end{theorem}

\proof Well, yes. \qed



\section{Filetypes used by LaTeX \label{sec:files}}

You will write your text as a .tex file using any text editor (though WYSIWYG 
editors are troublesome).
Traditionally one then runs \LaTeX\ and obtains a .dvi file, which
can be viewed on the screen using a dvi viewer. To include images, and then prepare the file for printing or submission,
one typically translates the .dvi into either .ps (Postscript) or .pdf (Adobe PDF).

Your report will be submitted to Stellar as a .pdf document.
The {\tt pdflatex} command
produces a .pdf file directly from a .tex file. This command works well with 
included .pdf files, but does not handle .eps files. 
An .eps file can be converted to a .pdf file by viewing it 
and saving as a .pdf file, or by {\tt ps2pdf filename.eps}, which produces
{\tt filename.pdf}.
Under MikTeX with WinEdt, all necessary commands will appear under ``Accessories'' in the WinEdt menu. 


%Under Linux or Unix for example, the commands are
%\begin{quote}
%{\tt latex filename.tex}
%\end{quote}
%which produces {\tt filename.dvi}, and then
%\begin{quote}
%{\tt dvips -f filename.dvi > filename.ps}
%\end{quote}
%followed by
%\begin{quote}
%{{\tt ps2pdf filename.ps}.}
%\end{quote}
%If the file does not contain embedded Postscript images,
%.pdf output can be obtained directly using
%\begin{quote}
%{{\tt pdflatex filename.tex}.}
%\end{quote}

Finally, Matlab can be made to produce .eps files by typing
\begin{quote}
{\tt print -deps filename}
\end{quote}
at the prompt. 



\section{Quoting sources\label{sec:bibliography}}

In your work, keep notes of the literature you've used, including websites. Cite the references you use; failure to do so constitutes plagiarism. Every bibliography item should be referenced somewhere in the paper.
Quote as precisely as possible: \cite[pages 76--78]{gurps} rather than \cite{gurps}. \cite{wildlife} was a useful background reference, too. 


\bibliographystyle{plain}
\begin{thebibliography}{9}
\bibitem{gurps}
Gurps, P., Care and feeding of maths professors. Cambridge Univ. Press, 2008.
\bibitem{wildlife}
Burps, X. Terrors and errors of project lab. {\em Journal of Wildlife and Conservation} 21 (2008), 112--134.
\end{thebibliography}

\section*{Appendix}
Appendices are useful for putting in code or data.

\end{document}

From Spring 2011 or before; revised by Miller, Spring 2013. 
